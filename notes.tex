\documentclass{article}
\usepackage[utf8]{inputenc}
\usepackage{array}
\usepackage{amsmath,amssymb}
\usepackage{booktabs}
\usepackage{caption}
\usepackage[nodayofweek]{datetime}
\usepackage{environ}
\usepackage{float}
\usepackage{enumitem}
\usepackage{fancyhdr}
\usepackage[landscape,margin=13mm,footskip=1pt,includefoot]{geometry}
\usepackage{graphicx}
\usepackage{hyperref}
\usepackage{multicol}
\usepackage{rotating}
\usepackage{tikz}
\usepackage{threeparttable}
\usepackage{url}
\usepackage{xspace}
\usepackage{amsmath,amssymb}
\usepackage{dsfont}
\usepackage{cancel}
\usepackage{graphicx}
\usepackage{xargs}
\usepackage{xspace}

% =============================================================================
%                                  Formatting
% =============================================================================

% Make a note on the margin.
\newcommand{\marnote}[1]{
  \reversemarginpar
  \marginpar[\raggedleft\footnotesize\textit{\\[3ex]#1}]%
      {\raggedright\footnotesize\textit{\\[3ex]#1}}
  \normalmarginpar
}

\newcommand{\pwiseii}[1]{\ensuremath{\left\{\begin{array}{ll}#1\end{array}}}
\newcommand{\pwiseiii}[1]{\ensuremath{\left\{\begin{array}{ll}#1\end{array}}}
\newcommand{\prn}[1]{\ensuremath{\left(#1\right)}}
\newcommand{\brk}[1]{\ensuremath{\left[#1\right]}}
\newcommand{\brc}[1]{\ensuremath{\left\{#1\right\}}}
\newcommand{\x}[1]{\ensuremath{\cancel{#1}}}

% =============================================================================
%                                 General Math
% =============================================================================

% Special functions and operators
\DeclareMathOperator{\erf}{erf}
\DeclareMathOperator{\logit}{logit}
\DeclareMathOperator{\sign}{sign}
\DeclareMathOperator*{\argmin}{\arg\!\min}

% Definitions
\def\define{:=}
\def\defined{=:}
\def\eqdef{\triangleq}

% Proofs
\def\qed{\ifhmode\unskip\nobreak\fi\hfill \ensuremath{\square}}

% Standard transformation function
\def\transform{\ensuremath{\varphi}\xspace}

% Logic
\newcommand{\comp}[1]{\neg{#1}}
\newcommand{\imp}{\ensuremath{\;\Longrightarrow\;}}
\newcommand{\pmi}{\ensuremath{\;\Longleftarrow\;}}
\newcommand{\nimp}{\ensuremath{\;\not\!\!\Longrightarrow\;}}
\newcommand{\npmi}{\ensuremath{\;\not\!\!\Longleftarrow\;}}
\newcommand{\eqv}{\ensuremath{\;\Longleftrightarrow\;}}

% Numbers.
\def\C{\mathbb{C}}
\def\N{\mathbb{N}}
\def\R{\mathbb{R}}
\def\Z{\mathbb{Z}}

% Matrices
\newcommand{\eyeii}{\ensuremath{\left(\begin{matrix}1 & 0 \\ 0 & 1\end{matrix}\right)}}
\newcommand{\eyeiii}{\ensuremath{\left(\begin{matrix}1 & 0 & 0 \\ 0 & 1 & 0 \\ 0 & 0 & 1\end{matrix}\right)}}

% Limits
\newcommand{\Lim}[2]{\ensuremath{\lim_{#1\to #2}}}
\newcommand{\limx}[1][\infty]{\ensuremath{\lim_{x\to #1}}}
\newcommand{\limn}[1][\infty]{\ensuremath{\lim_{n\to #1}}}

% Sums and products
\newcommand{\Sum}[2][i=1]{\ensuremath{\sum_{#1}^{#2}}}
\newcommand{\sumin}{\ensuremath{\sum_{i=1}^n}}
\newcommand{\sumiN}{\ensuremath{\sum_{i=1}^N}}
\newcommand{\sumim}{\ensuremath{\sum_{i=1}^m}}
\newcommand{\sumjk}{\ensuremath{\sum_{j=1}^k}}
\newcommand{\sumjn}{\ensuremath{\sum_{j=1}^n}}
\newcommand{\sumjm}{\ensuremath{\sum_{j=1}^m}}
\newcommand{\isum}[1][n]{\ensuremath{\sum_{#1}^\infty}}
\newcommand{\dsum}[4][i=1]{\ensuremath{\sum_{#1}^{#2}\sum_{#3}^{#4}}}
\newcommand{\Prod}[2][i=1]{\ensuremath{\prod_{#1}^{#2}}}
\newcommand{\prodin}{\ensuremath{\prod_{i=1}^n}}
\newcommand{\prodjn}{\ensuremath{\prod_{j=1}^n}}

% Derivatives
\newcommand{\der}[2][]{\ensuremath{\frac{d #1}{d #2}}}
\newcommand{\dder}[2][]{\ensuremath{\frac{d^2 #1}{d #2^2}}}
\newcommand{\pder}[2][]{\ensuremath{\frac{\partial #1}{\partial #2}}}
\newcommand{\pdder}[2][]{\ensuremath{\frac{\partial^2 #1}{\partial #2^2}}}
\newcommand{\mpder}[3][]{%
  \ensuremath{\frac{\partial^2 #1}{\partial #2 \partial #3}}}

% Differentials
%\renewcommand{\d}[1]{\,\mathrm{d}#1}
\renewcommand{\d}[1]{\,d#1}
\def\ds{\d{s}}
\def\dt{\d{t}}
\def\dtheta{\d{\theta}}
\def\du{\d{u}}
\def\dx{\d{x}}
\def\dy{\d{y}}
\def\dfx{\d{F_X(x)}}
\def\dfy{\d{F_Y(y)}}
\def\dfhatx{\d{\widehat{F}_n(x)}}

% Transcendentals w/ extended arguments.
\newcommand{\Exp}[1]{\ensuremath{\exp\left\{#1\right\}}}
\newcommand{\Log}[1]{\ensuremath{\log\left\{#1\right\}}}

% =============================================================================
%                          Probability and Statistics
% =============================================================================

% Formatted terminology.
\def\bias{\textsf{bias}\xspace}
\def\se{\textsf{se}\xspace}
\def\pdf{\textsc{pdf}\xspace}
\def\cdf{\textsc{cdf}\xspace}
\def\ise{\textsc{ise}\xspace}
\def\pgf{\textsc{pgf}\xspace}
\def\mgf{\textsc{mgf}\xspace}
\def\mse{\textsc{mse}\xspace}
\def\mspe{\textsc{mspe}\xspace}
\def\mle{\textsc{mle}\xspace}
\def\mom{\textsc{mom}\xspace}
\def\are{\textsc{are}\xspace}
\def\rss{\textsc{rss}\xspace}
\def\ess{\textsc{ess}\xspace}
\def\tss{\textsc{tss}\xspace}

% Naming shortcuts.
\def\ahat{\ensuremath{\widehat{\alpha}}}
\def\atil{\ensuremath{\tilde{\alpha}}}
\def\bhat{\ensuremath{\widehat{\beta}}}
\def\btil{\ensuremath{\tilde{\beta}}}
\def\dhat{\ensuremath{\widehat{\delta}}}
\def\ehat{\ensuremath{\hat{\epsilon}}}
\def\ghat{\ensuremath{\widehat{\gamma}}}
\def\khat{\ensuremath{\widehat{\kappa}}}
\def\lhat{\ensuremath{\widehat{\lambda}}}
\def\ltil{\ensuremath{\tilde{\lambda}}}
\def\mhat{\ensuremath{\widehat{\mu}}}
\def\nhat{\ensuremath{\widehat{\nu}}}
\def\mtil{\ensuremath{\tilde{\mu}}}
\def\psihat{\ensuremath{\widehat{\psi}}}
\def\shat{\ensuremath{\widehat{\sigma}}}
\def\stil{\ensuremath{\tilde{\sigma}}}
\def\that{\ensuremath{\widehat{\theta}}}
\def\ttil{\ensuremath{\widetilde{\theta}}}
\def\rhohat{\widehat{\rho}}
\def\xihat{\widehat{\xi}}

\def\sehat{\ensuremath{\widehat{\se}}}
\def\fhat{\ensuremath{\widehat{f}}}
\def\Fhat{\ensuremath{\widehat{F}}}
\def\fnhat{\ensuremath{\widehat{f}_n}}
\def\Fnhat{\ensuremath{\widehat{F}_n}}
\def\Jhat{\ensuremath{\widehat{J}}}
\def\phat{\ensuremath{\widehat{p}}}
\def\ptil{\ensuremath{\tilde{p}}}
\def\rhat{\widehat{r}}
\def\Rbar{\bar{R}}
\def\Rhat{\widehat{R}}
\def\Qbar{\bar{Q}}
\def\Qhat{\widehat{Q}}
\def\Xhat{\widehat{X}}
\def\xbar{\bar{x}}
\def\Xbar{\bar{X}}
\def\Xsqbar{\overline{X^2}}
\def\xnbar{\overline{x}_n}
\def\Xnbar{\overline{X}_n}
\def\Yhat{\widehat{Y}}
\def\ybar{\overline{y}}
\def\Ybar{\overline{Y}}
\def\Ynbar{\overline{Y}_n}

% Random variables.
\def\rv{\textsc{rv}\xspace}
\def\iid{\ensuremath{\textsc{iid}}\xspace}
\def\dist{\ensuremath{\sim}\xspace}
\def\disteq{\ensuremath{\stackrel{D}{=}}\xspace}
\def\distiid{\ensuremath{\stackrel{iid}{\sim}}\xspace}
\def\ind{\ensuremath{\perp\!\!\!\perp}\xspace}
\def\nind{\ensuremath{\perp\!\!\!\!\big\vert\!\!\!\!\perp}\xspace}
\def\Xon{\ensuremath{X_1,\dots,X_n}\xspace}
\def\xon{\ensuremath{x_1,\dots,x_n}\xspace}
\def\giv{\ensuremath{\,|\,}}
\def\Giv{\ensuremath{\,\big|\,}}
\def\GIV{\ensuremath{\,\Big|\,}}
\newcommand{\indicator}[1]{\mathds{1}_{\left\{#1\right\}}}

% Probability, expectation, and variance.
\def\prob{\mathbb{P}}
\renewcommand{\Pr}[2][]{\ensuremath{\prob_{#1}\left[#2\right]}\xspace}
\newcommand{\E}[2][]{\ensuremath{\mathbb{E}_{#1}\left[#2\right]}}
\newcommand{\V}[2][]{\ensuremath{\mathbb{V}_{#1}\left[#2\right]}}
\newcommand{\cov}[2][]{\ensuremath{\mathrm{Cov}_{#1}\left[#2\right]}}
\newcommand{\corr}[2][]{\ensuremath{\rho_{#1}\left[#2\right]}}
\def\sd{\ensuremath{\textsf{sd}}\xspace}
\def\samplemean{\ensuremath{\bar{X}_n}\xspace}
\def\samplevar{\ensuremath{S^2}\xspace}
\def\za{\ensuremath{z_{\alpha}}}
\def\zat{\ensuremath{z_{\alpha/2}}}

% Inference
\def\Ll{\ensuremath{\mathcal{L}}\xspace}
\def\Lln{\ensuremath{\Ll_n}\xspace}
\def\ll{\ensuremath{\ell}}
\def\lln{\ensuremath{\ll_n}}

% Hypothesis testing
\newcommand{\hyp}[2]{
\ensuremath{H_0:#1 \ifhmode\quad\text{versus}\quad\fi\text{ vs. } H_1:#2}}

% Convergence.
\def\conv{\rightarrow}
\def\convinf{\rightarrow_{n\to\infty}}
\def\pconv{\stackrel{\text{\tiny{P}}}{\rightarrow}}
\def\npconv{\stackrel{\text{\tiny{P}}}{\nrightarrow}}
\def\dconv{\stackrel{\text{\tiny{D}}}{\rightarrow}}
\def\ndconv{\stackrel{\text{\tiny{D}}}{\nrightarrow}}
\def\qmconv{\stackrel{\text{\tiny{qm}}}{\rightarrow}}
\def\nqmconv{\stackrel{\text{\tiny{qm}}}{\nrightarrow}}
\def\asconv{\stackrel{\text{\tiny{as}}}{\rightarrow}}
\def\nasconv{\stackrel{\text{\tiny{as}}}{\nrightarrow}}

%
% Distributions
%
\newcommandx{\unif}[1][1={a,b}]{\textrm{Unif}\left({#1}\right)}
\newcommandx{\unifd}[1][1={a,\ldots,b}]{\textrm{Unif}\left\{{#1}\right\}}
\newcommandx{\dunif}[3][1=x,2=a,3=b]{\frac{I(#2<#1<#3)}{#3-#2}}
\newcommandx{\dunifd}[3][1=x,2=a,3=b]{\frac{I(#2\le#1\le#3)}{#3-#2+1}}
\newcommandx{\punif}[3][1=x,2=a,3=b]{
\begin{cases} 0 & #1 < #2 \\ \frac{#1-#2}{#3-#2} & #2 < #1 < #3 \\ 1 & #1 > #3\\\end{cases}}
\newcommandx{\punifd}[3][1=x,2=a,3=b]{
\begin{cases} 0 & #1 < #2\\ \frac{\lfloor#1\rfloor-#2+1}{#3-#2} & #2 \le #1 \le #3 \\ 1 & #1 > #3\\ \end{cases}}

% Bernoulli
\newcommandx\bern[1][1=p]{\textrm{Bern}\left({#1}\right)}
\newcommandx\dbern[2][1=x,2=p]{#2^{#1} \left(1-#2\right)^{1-#1}}
\newcommandx\pbern[2][1=x,2=p]{\left(1-#2\right)^{1-#1}}

% Binomial
\newcommandx\bin[1][1={n,p}]{\textrm{Bin}\left(#1\right)}
\newcommandx\dbin[3][1=x,2=n,3=p]{\binom{#2}{#1}#3^#1\left(1-#3\right)^{#2-#1}}

% Multinomial
\newcommandx\mult[1][1={n,p}]{\textrm{Mult}\left(#1\right)}
\newcommandx\dmult[3][1=x,2=n,3=p]{\frac{#2!}{#1_1!\ldots#1_k!}#3_1^{#1_1}\cdots#3_k^{#1_k}}

% Hypergeometric
\newcommandx\hyper[1][1={N,m,n}]{\textrm{Hyp}\left({#1}\right)}
\newcommandx\dhyper[4][1=x,2=N,3=m,4=n]{\frac{\binom{#3}{#1}\binom{#2-#3}{#4-#1}}{\binom{#2}{#4}}}

% Negative Binomial
\newcommandx\nbin[1][1={r,p}]{\textrm{NBin}\left({#1}\right)}
\newcommandx\dnbin[3][1=x,2=r,3=p]{\binom{#1+#2-1}{#2-1}#3^#2(1-#3)^#1}
\newcommandx\pnbin[3][1=x,2=r,3=p]{I_#3(#2,#1+1)}

% Geometric
\newcommandx\geo[1][1=p]{\textrm{Geo}\left(#1\right)}
\newcommandx\dgeo[2][1=x,2=p]{#2(1-#2)^{#1-1}}
\newcommandx\pgeo[2][1=x,2=p]{1-(1-#2)^#1}

% Poisson
\newcommandx\pois[1][1=\lambda]{\textrm{Po}\left({#1}\right)}
\newcommandx\dpois[2][1=x,2=\lambda]{\frac{#2^#1 e^{-#2}}{#1!}}
\newcommandx\ppois[2][1=x,2=\lambda]{e^{-#2}\sum_{i=0}^#1\frac{#2^i}{i!}}

% Normal
\newcommandx\norm[1][1={\mu,\sigma^2}]{\mathcal{N}\left({#1}\right)}
\newcommandx\dnorm[3][1=x,2=\mu,3=\sigma]%
{\frac{1}{#3\sqrt{2\pi}}\Exp{-\frac{\left(#1-#2\right)^2}{2 #3^2}}}
\newcommandx\pnorm[1][1=x]{\Phi\left({#1}\right)}
\newcommandx\qnorm[1]{\Phi^{-1}\left({#1}\right)}

% Multivariate Normal
\newcommandx\mvn[1][1={\mu,\Sigma}]{\mathrm{MVN}\left({#1}\right)}

% Exponential
\newcommandx\ex[1][1=\beta]{\textrm{Exp}\left(#1\right)}
\newcommandx\dex[2][1=x,2=\beta]{\frac{1}{#2}e^{-#1/#2}}
\newcommandx\pex[2][1=x,2=\beta]{1-e^{-#1/#2}}

% Gamma
\newcommandx\gam[1][1={\alpha,\beta}]{\textrm{Gamma}\left({#1}\right)}
\newcommandx\dgamma[3][1=x,2=\alpha,3=\beta]%
{\frac{#3^{#2}}{\Gamma\left( #2 \right)} #1^{#2-1}e^{-#3#1}}

% InverseGamma
\newcommandx\invgamma[1][1={\alpha,\beta}]{\textrm{InvGamma}\left({#1}\right)}
\newcommandx\dinvgamma[3][1=x,2=\alpha,3=\beta]%
{\frac{#3^{#2}}{\Gamma\left(#2\right)}#1^{-#2-1}e^{-#3/#1}}
\newcommandx\pinvgamma[3][1=x,2=\alpha,3=\beta]%
{\frac{\Gamma\left(#2,\frac{#3}{#1}\right)}{\Gamma\left(#2\right)}}

% Beta
\newcommandx\bet[1][1={\alpha,\beta}]{\textrm{Beta}\left(#1\right)}
\newcommandx\dbeta[3][1=x,2=\alpha,3=\beta]
{\frac{\Gamma\left(#2+#3\right)}{\Gamma\left(#2\right)\Gamma\left(#3\right)}#1^{#2-1}\left(1-#1\right)^{#3-1}}

% Dirichlet
\newcommandx\dir[1][1={\alpha}]{\textrm{Dir}\left(#1\right)}
\newcommandx\ddir[3][1=x,2=\alpha]{\frac{\Gamma\left(\sum_{i=1}^k #2_i\right)}{\prod_{i=1}^k\Gamma\left(#2_i\right)}\prod_{i=1}^k #1_i^{#2_i-1}}

% Weibull
\newcommandx\weibull[1][1={\alpha}]{\textrm{Dir}\left(#1\right)}
\newcommandx\dweibull[3][1=x,2=\lambda,3=k]{\frac{#3}{#2}
\left(\frac{#1}{#2}\right)^{#3-1} e^{-(#1/#2)^k}}

% Chi-squard
\newcommandx\chisq[1][1=k]{\chi_{#1}^2}

% Zeta
\newcommandx\zet[1][1=s]{\textrm{Zeta}\left(#1\right)}
\newcommandx\dzeta[2][1=x,2=s]{\frac{#1^{-#2}}{\zeta\left(#2\right)}}

% Time Series
\newcommandx\AR[1][1=p]{\mathsf{AR}\left({#1}\right)}
\newcommandx\MA[1][1=q]{\mathsf{MA}\left({#1}\right)}
\newcommandx\ARMA[1][1={p,q}]{\mathsf{ARMA}\left({#1}\right)}
\newcommandx\ARIMA[1][1={p,d,q}]{\mathsf{ARIMA}\left({#1}\right)}
\newcommandx\SARIMA[3][1={p,d,q},2={P,D,Q},3=s]{\mathsf{ARIMA}\left(#1\right) \times \left(#2\right)_{#3}}


% =============================================================================
%                                 Algorithms
% =============================================================================

\newcommandx\step[1][1=t]{^{(#1)}}

\usepackage{bm}
\usepackage{lipsum}


\hypersetup{
    colorlinks=true,
    urlcolor=blue}
% sources:
% https://github.com/mavam/stat-cookbook
% http://statweb.stanford.edu/~kriss1/statistics-consulting-cheat.pdf
% wikipedia
% goodfellow et al Deep Learning
%

\begin{document}

\begin{multicols*}{2}
\section{Probability and statistics}


\subsection{Working with probability distributions}
\begin{itemize}
    \item Given probability distribution $\mathbb{P}$, sample space $\Omega$, and event $A \subseteq \Omega$:
    \begin{itemize}
        \item $\mathbb{P} \geq 0 \quad \forall A$ (probabilities are nonzero)
        \item $\mathbb{P}[\Omega] = 1$ (probabilities sum to 1)
        \item $\mathbb{P}[\varnothing] = 0$ (probability of empty set is 0)
        \item $\Pr{\displaystyle\bigsqcup_{i=1}^\infty A_i}
        = \displaystyle\sum_{i=1}^\infty \Pr{A_i} = 1$
    \end{itemize}
    \item Probabilities are independent when the joint probability is equal to the product of the marginal probabilities.
    \[A \ind B \eqv \Pr{A \cap B} = \Pr{A}\Pr{B}\]

    \item The conditional probability of $A$ given $B$ is the joint probability of $A$ and $B$ divided by the probability of just $B$.
    \[\Pr{A \giv B} = \frac{\Pr{A \cap B}}{\Pr{B}}\]

    \item The Probability Mass Function (PMF) is used to describe the behavior of \emph{discrete} probability distributions.
    \[f_X(x) = \Pr{X = x}\]

    \item The Probability Density Function (PDF) is the equivalent for \emph{continuous} distributions. We use the PDF to determine the probability that random variable $X$ is between $A$ and $B$.
    \[\Pr{a \le X \le b} = \int_a^b f(x)\dx\]

    \item The Cumulative Distribution Function (CDF) is the integral of the PDF and we use it to determine the probability that random variable $X$ is less than or equal to $x$. It maps $\R \to [0,1]$ and is monotonically non-decreasing. The left and right limits are $0$ and $1$ ($\lim_{x\to -\infty} = 0$ and $\lim_{x\to \infty} = 1$).
    \[F_X(x) = \Pr{X \le x}\]

\end{itemize}

\subsubsection{Notes on the normal distribution}
\begin{itemize}
    \item The normal distribution is a function of mean $\mu$ and variance $\sigma^2$
    \item The simplest case is the \textbf{standard normal distribution}, $Z \sim \mathcal{N}(0, 1)$, which reduces to:
    $$\phi(x) = \frac{1}{\sqrt{2\pi}}e^{-\frac{1}{2}x^2}$$

    \begin{itemize}
        \item Interestingly, others have defined even simpler standard normals. Gauss proposed $\sigma^2 = \frac{1}{2}$, which reduces to:
        $$\phi(x) = \frac{e^{-x^2}}{\sqrt{pi}}$$

        \item Stigler proposed a formulation with $\sigma^2 = \frac{1}{2\pi}$, leading to:
        $$\phi(x) = e^{-\pi x^2}$$

    \end{itemize}
    \item We can convert any normally distributed variable $X$ to a \emph{standard normal} by subtracting the mean and dividing by the standard deviation.
    $$ Z = \frac{X - \mu}{\sigma} $$
    \item \textbf{68-95-99.7 rule:} the percentage of values that lie within 1, 2, and 3 standard deviations of the mean of a normal distribution are $68.27\%$, $95.45\%$, and $99.73\%$ respectively. A $\mu \pm 3\sigma$ deviation should occur at a frequency of about 1 in 370.
    \item The Gauss Error Function gives the probability of a RV $Z \sim \mathcal{N}(0, 1/2)$ falling in the range $[-x, x]$:
    $$ \mathrm{erf}(x) = \frac{2}{\sqrt{\pi}}\int_{0}^{x}e^{-t^2}$$

\end{itemize}

\subsubsection{Notes on the uniform distribution}
\begin{itemize}
    \item The continuous uniform distribution is a function of the minimum and maximum values  $a$ and $b$ with mean and median equal to $\frac{a+b}{2}$
    \item The \textbf{standard uniform} is a random variable $\sim \mathcal{U}(0, 1)$
    \item The PDF of a uniform distribution is a horizontal line from $a$ to $b$
    %\item 
\end{itemize}

\subsubsection{Notes on binomial distribution}
\begin{itemize}
    \item Discrete distribution $\mathcal{B}(n, p)$ for the number of successes in a sequence of $n$ Bernoulli trials with probability of success $p$.
    \item 
    \item 
\end{itemize}

%TODO: would be pretty sick to have a summary of inverse transform sampling here

%TODO: section on binomial distribution, cheat sheet for permutations / combinations

%TODO: pick out the good parts of kolmogorov's text on probability 

\end{multicols*}

\subsection{Common distributions}

\begin{center}
\small
\begin{tabular}{@{}l*6{>{\begin{math}\displaystyle}c<{\end{math}}}@{}}
  \toprule &&&&&& \\[-2ex]
  & \text{Type}
  & F_X(x) & f_X(x) & \E{X} & \V{X} & M_X(s) \\[1ex]

  \midrule

  Uniform & Discrete & \punifd & \dunifd &
  \frac{a+b}{2} & \frac{(b-a+1)^2-1}{12} &
  \frac{e^{as}-e^{-(b+1)s}}{s(b-a)} \\[3ex]

  Bernoulli & Discrete & \pbern & \dbern &
  p & p(1-p) &
  1-p+pe^s \\[3ex]

  Binomial & Discrete & I_{1-p}(n-x,x+1) & \dbin &
  np & np(1-p) &
  (1-p+pe^s)^n \\[3ex]

  Multinomial & Discrete & & \dmult \quad \sum_{i=1}^k x_i = n&
  \left( {\begin{array}{*{20}{c}}
    {n{p_1}}\\
    \vdots \\
    {n{p_k}}
  \end{array}} \right) & \left( {\begin{array}{*{20}{c}}
    {n{p_1}(1 - {p_1})}&{ - n{p_1}{p_2}}\\
    { - n{p_2}{p_1}}& \ddots
    \end{array}} \right) &
  \left( \sum_{i=0}^k p_i e^{s_i} \right)^n \\[3ex]


  Poisson & Discrete & \ppois & \dpois &
  \lambda & \lambda &
  e^{\lambda(e^s-1)}\\[3ex]

 Uniform & Continuous & \punif & \dunif &
  \frac{a+b}{2} & \frac{(b-a)^2}{12} &
  \frac{e^{sb}-e^{sa}}{s(b-a)} \\[3ex]

  Normal & Continuous &
  \Phi(x)=\displaystyle\int_{-\infty}^x \phi(t)\,dt &
  \phi(x)=\dnorm &
  \mu & \sigma^2 &
  \Exp{\mu s + \frac{\sigma^2s^2}{2}}\\[3ex]

  Log-Normal & Continuous &
  \frac{1}{2}+\frac{1}{2} \erf\left[\frac{\ln x-\mu}{\sqrt{2\sigma^2}}\right] &
  \frac{1}{x\sqrt{2\pi\sigma^2}} \Exp{-\frac{(\ln x - \mu)^2}{2\sigma^2}} &
  e^{\mu+\sigma^2/2} &
  (e^{\sigma^2}-1) e^{2\mu+\sigma^2} &
  \\[3ex]

  Multivariate Normal & Continuous & &
  (2\pi)^{-k/2} |\Sigma|^{-1/2} e^{-\frac{1}{2}(x-\mu)^T \Sigma^{-1}(x-\mu)} &
  \mu & \Sigma &
  \Exp{\mu^T s + \frac{1}{2} s^T \Sigma s}\\[3ex]

  Student's $t$ & Continuous
  & I_x\left( \frac{\nu}{2},\frac{\nu}{2} \right)
  & \frac{\Gamma\left(\frac{\nu+1}{2}\right)}
    {\sqrt{\nu\pi}\Gamma\left(\frac{\nu}{2}\right)}
    \left(1+\frac{x^2}{\nu}\right)^{-(\nu+1)/2}
  & 0 \quad \nu  > 1
  & \begin{cases}
      \displaystyle\frac{\nu}{\nu-2} & \nu > 2 \\
      \infty & 1 < \nu \le 2
    \end{cases}
  & \\[3ex]

  Chi-square & Continuous &
  \frac{1}{\Gamma(k/2)} \gamma\left(\frac{k}{2}, \frac{x}{2}\right) &
  \frac{1}{2^{k/2} \Gamma(k/2)} x^{k/2-1} e^{-x/2}&
  k & 2k &
  (1-2s)^{-k/2} \; s<1/2\\[3ex]

  Exponential\tnote{$\ast$} & Continuous & \pex & \dex &
  \beta & \beta^2 &
  \frac{1}{1-\frac{s}{\beta}} \left(s<\beta\right) \\[3ex]

\bottomrule
\end{tabular}
\end{center}
\clearpage
\begin{multicols*}{2}


\subsection{Hypothesis testing}
\begin{itemize}
    \item Framework for filtering implausible scientific claims
    \item Basic steps:
    \begin{enumerate}
        \item State relevant null hypothesis ($H_0$) and alternative hypothesis ($H_1$)
        \begin{itemize}
            \item Two-sided: $H_0: \theta = \theta_0$ vs $H_1: \theta \neq \theta_0$
            \item One-sided: $H_0: \theta \leq \theta_0$ vs $H_1: \theta > \theta_0$
        \end{itemize}
        \item Determine relevant test statistic ($T$) distribution, typically Student's $t$ or normal distribution
        \item Select significance level ($\alpha$, often 5\% or 1\%)
        \item Calculate rejection region (critical region), which contains all values of $x$ for which $T(x)$ is greater than the critical value $c$: $R = \{x: T(x) > c\}$
        \item Determine whether to accept or reject $H_0$
    \end{enumerate}
    \item Alternatively, just calculate the $p$-value (probability given $H_0$ of getting a result at least as extreme as that which was observed). Reject the null hypothesis if $p \leq \alpha$.
    \item Common ranges for $p$-values are:
    \begin{itemize}
        \item $< 0.01$: very strong evidence against $H_0$
        \item $[0.01, 0.05]$: strong evidence against $H_0$
        \item $[0.05, 0.10]$: weak evidence against $H_0$
        \item $> 0.1$: yikes man
    \end{itemize}
    \item Type I errors (false positives) occur when we incorrectly \textbf{reject} the null hypothesis. This is equivalent to $\alpha$.
    \item Type II errors (false negatives) occur when we incorrectly \textbf{fail to reject} the null hypothesis.
\begin{center}
\begin{tabular}{l|cc}
  & \textsf{Retain} $H_0$ & \textsf{Reject} $H_0$ \\
  \hline
  $H_0$ \textsf{true} & $\surd$ & Type I Error ($\alpha$)\\
  $H_1$ \textsf{true} & Type II Error ($\beta$) &
  $\surd$ (power) \\
\end{tabular}
\end{center}
%TODO: multiple comparisons comment
\end{itemize}

\subsection{Bayesian inference}
%TODO: complete basic outline

\section{Linear algebra}
%% note- yoinked from Goodfellow et al Deep Learning book
\subsection{Objects and notation}
\begin{itemize}
    \item Let scalar $s \in \mathbb{R}$
    \item Let vector $\bm{x} \in \mathbb{R}^n$. We should assume that all vectors are `column vectors' (ie a matrix in $\mathbb{R}^{n \times 1}$)
    \item Let 2-d matrix $\bm{A} \in \mathbb{R}^{m \times n}$. We'll identify specific elements like this:
    $$\begin{bmatrix}
    A_{1,1} & A_{1,2} \\
    A_{2,1} & A_{2,2} \\
    \end{bmatrix}$$
    \begin{itemize}
        \item We'll denote a whole column $i$ of a matrix as $\bm{A}_{:,i}$ and a row $j$ as $\bm{A}_{j, :}$
    \end{itemize}
    \item Tensors extend beyond 2d, eg: $\textbf{A}_{i,j,k}$
\end{itemize}
\subsection{Basic matrix operations review}
\begin{itemize}
    \item The \textbf{transpose} operation mirrors the matrix across the diagonal and is denoted $\bm{A}^\mathrm{T}$.
    $$
    \bm{A} = \begin{bmatrix}
    A_{1,1} & A_{1,2} \\
    A_{2,1} & A_{2,2} \\
    A_{3,1} & A_{3,2} \\
    \end{bmatrix} \Rightarrow
    \bm{A}^\mathrm{T} = \begin{bmatrix}
    A_{1,1} & A_{2,1} & A_{3, 1} \\
    A_{1,2} & A_{2,2}, &  A_{3, 2}\\
    \end{bmatrix}
    $$
    \item Addition of matrices is element-wise, and therefore requires them to be the same shape.
    $$ C_{i,j} = A_{i,j} + B_{i,j} \qquad \{A, B, C\} \in \mathbb{R}^{m \times n} $$
    \item The \textbf{matrix product} of $\bm{A} \in \mathbb{R}^{m \times n}$ and $\bm{B} \in \mathbb{R}^{n \times p}$ is $\bm{C} \in \mathbb{R}^{m \times p}$. Note that the number of columns in the first matrix must be equal to the number of rows in the second matrix ($m$). Each element in $\bm{C}_{i,j}$ can be thought of as the dot product between row $i$ of $\bm{A}$ and column $j$ of $\bm{B}$.
    $$ C_{i,j} = \sum_{k} A_{i,k}B_{k,j}$$
    \item Some matrix operation properties:
    \begin{itemize}
        \item Distributive: $\bm{A}(\bm{B}+\bm{C}) = \bm{AB}+\bm{AC}$
        \item Associative: $\bm{A}(\bm{BC}) = (\bm{AB})\bm{C}$
        \item \textbf{NOT} commutative: $\bm{AB} \neq \bm{BA}$
        \item Transpose product: $(\bm{AB})^\mathrm{T} = \bm{B}^\mathrm{T}\bm{A}^\mathrm{T}$
    \end{itemize}
\end{itemize}

\subsubsection{The identity matrix}
\begin{itemize}
    \item We'll define the \textbf{identity matrix} $\bm{I}_n$ as the matrix that does not change a vector $\bm{x}$ of dimension $n$ when they are multiplied together so that $\forall \bm{x} \in \mathbb{R}^n, \quad \bm{I}_n\bm{x}=\bm{x}$. The identity matrix is just a square matrix with $1$ on the diagonal and $0$ elsewhere, so for $\bm{x} \in \mathbb{R}^3$:
     $$ \bm{I}_3 = \begin{bmatrix}
    1 & 0 & 0 \\
    0 & 1 & 0 \\
    0 & 0 & 1 \\ \end{bmatrix} $$\end{itemize}


\subsubsection{Matrix inversion}
\begin{itemize}
    \item The \textbf{matrix inverse} of $\bm{A}$ is denoted $\bm{A}^{-1}$ and we define it such that:
    $$ \bm{A}^{-1}\bm{A} = \bm{I}_n$$
    \item $\bm{A}$ is \textbf{invertible} if it is square ($\in \mathbb{R}^{n \times n}$) and non-singular.
    \begin{itemize}
        \item A square matrix is \textbf{singular} $\iff$  it has a determinant of $0$
        \item Singular matrices have linearly dependent columns
        \begin{itemize}
            \item The \textbf{determinant} of a matrix (usually denoted $\mathrm{det}(\bm{A})$ or $\vert \bm{A}\vert$) is a scalar factor that can be computed from the elements of a square matrix. For a $2\times 2$ matrix:
            $$ \bm{A} =
            \begin{bmatrix}
            a & b \\
            c & d \\ \end{bmatrix}
            \Rightarrow \vert \bm{A} \vert = ad-bc $$
            \end{itemize}
\end{itemize}
    \item For other important properties of invertible matrices see \href{https://en.wikipedia.org/wiki/Invertible_matrix#The_invertible_matrix_theorem}
    {\textbf{Wikipedia: Invertible matrix theorem}}
\end{itemize}
\subsection{Systems of linear equations}
\begin{itemize}
    \item We can define a \textbf{system of linear equations}, $\bm{Ax}=\bm{b}$. $\bm{A}$ is a known matrix of coefficients, $\bm{b}$ is a known vector, and we're trying to solve for vector $\bm{x}$. The matrix $\bm{A} \in \mathbb{R}^{m \times n}$ describes a system of $m$ equations with $n$ unknowns.
    \item This is really the same as writing:
    $$ x_1 \begin{bmatrix}
     a_{11}\\
     a_{21}\\
     \vdots \\
     a_{m1}\\ \end{bmatrix} +
     x_2 \begin{bmatrix}
     a_{12}\\
     a_{22}\\
     \vdots \\
     a_{m2}\\ \end{bmatrix} +
     \hdots +
     x_n \begin{bmatrix}
     a_{1n}\\
     a_{2n}\\
     \vdots \\
     a_{mn}\\ \end{bmatrix} =
     \begin{bmatrix}
     b_{1}\\
     b_{2}\\
     \vdots \\
     b_{n}\\ \end{bmatrix}
     $$

    \end{itemize}

\section{Differential equations}
\subsection{Calculus refresher}
\begin{itemize}
    \item Some useful properties / rules with differentiable functions $f(x)$ and $g(x)$:
    \begin{itemize}
        \item $(cf)' = c(f')$ for any constant $c$
        \item $c' = 0$ for any constant $c$
        \item $(f+g)' = f' + g'$
        \item \textbf{Power rule:} $(x^n)' = nx^{n-1}$
        \item \textbf{Product rule:} $(fg)' = f'g + g'f$
        \item \textbf{Quotient rule:} $(\frac{f}{g})' = \frac{f'g - g'f}{g^2}$
        \item \textbf{Chain rule:} $f(g(x))' = f'(g)g'$
    \end{itemize}
    \item Common derivatives:
    \begin{itemize}
        \item $\frac{d}{dx}\; x = 1$
        \item $\frac{d}{dx}\; cx = c$
        \item $\frac{d}{dx}\; e^x = e^x$
        \item $\frac{d}{dx}\; \ln{x} = \frac{1}{x}, \quad x > 0$
        \item $\frac{d}{dx}\; \ln{\lvert x \rvert}= \frac{1}{x}, \quad x \neq 0$
        \item $\frac{d}{dx}\; c^x = c^x \ln{c}$
        \item $\frac{d}{dx}\; \sin{x} = \cos{x}$
        \item $\frac{d}{dx}\; \cos{x} = -\sin{x}$
        \item $\frac{d}{dx}\; \tan{x} = \sec^2{x}$
        \item $\frac{d}{dx}\; \log_c{x}=\frac{1}{x\ln{c}}, \quad x > 0 $
    \end{itemize}
    \item Common antiderivatives:
    \begin{itemize}
        \item $\int 0 \;dx = C$
        \item $\int 1 \;dx = x + C$
        \item $\int n \;dx = nx + C$
        \item $\int e^x \;dx = e^x + C$
        \item $\int \frac{1}{x} \;dx = \ln{x} +C$
        \item $\int x^n \;dx = \frac{x^{n+1}}{n+1} + C, \quad n \neq -1$
        \item $\int \sin{x} \; dx = -\cos{x} + C$
        \item $\int \cos{x} \; dx = \sin{x} + C$
    \end{itemize}
\item Fundamental theorem of calculus:
    $$ \int_a^b\frac{dy}{dx}dx=y(b)-y(a) \quad \iff \quad \frac{d}{dx}\int_a^x f(s)ds = f(x) $$
\item Three ways to use the fact that $\frac{dy}{dx} \approx \frac{\Delta y}{\Delta x}$
\begin{enumerate}[label=\alph*.]
    \item knowing $\Delta x$ and $dy/dx$, we know $\Delta y \approx \Delta x \frac{dy}{dx}$ (linear approximation)
    \item knowing $\Delta y$ and $dy/dx$, we know $\Delta x \approx \frac{\Delta y}{dy/dx}$ (Newton's method)
    \item approximate the derivative if we know $\Delta y$ and $\Delta x$ because $dy/dx \approx \frac{\Delta y}{\Delta x}$
     \begin{itemize}
         \item \emph{note: better to take a centered difference (half step each way)}
             \[ \frac{dy}{dx} \approx \frac{y(x+\frac{1}{2}\Delta x) - y(x-\frac{1}{2} \Delta x)}{\Delta x}\]
 \end{itemize}
\end{enumerate}
\item Taylor series: allows us to predict $y(x)$ from derivatives at $x = x_0$
    \[ y(x_0 + \Delta x) = y_0 + (\Delta x)y'_0 + \cdots  + \frac{1}{n!}(\Delta x)^n y_0^{(n)} \] 
    \[ = \sum_{n=0}^{\infty} \frac{(\Delta x)^n}{n!}y^{(n)}(x_0) \]
\end{itemize}

\subsection{1st order differential equations}
\subsubsection{}

\subsubsection{}


\subsection{2nd order differential equations}


\end{multicols*}
\end{document}
